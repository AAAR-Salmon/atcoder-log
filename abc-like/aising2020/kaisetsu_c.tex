\documentclass[11pt, dvipdfmx]{jarticle}
\usepackage{amsmath, amssymb}
\usepackage[top=2.5cm, bottom=2.5cm, left=2.5cm, right=2.5cm, includefoot, dvipdfmx]{geometry}
\usepackage{ascmac}
\usepackage{verbatim}

\newcommand*{\set}[1]{\{#1\}}

\title{エイシング プログラミング コンテスト 2020:C のフランクな解説}
\author{終に鮭}

\begin{document}

\maketitle

\section*{TL;DR}

\begin{itemize}
	\item 単純に for と if でカウントするだけ
	\item 普通に全探索するともちろんTLEだが,ちょっと枝刈りすれば耐える
\end{itemize}

\section*{略解}

適当に全探索で組んでみよう.
\begin{verbatim}
     1  #include <stdio.h>
     2
     3  int main() {
     4    int N;
     5    scanf("%d", &N);
     6    for (int n = 1; n <= N; n++) {
     7      int cnt = 0;
     8      for (int i = 1; i <= 100; i++) {
     9        for (int j = 1; j <= 100; j++) {
    10          for (int k = 1; k <= 100; k++) {
    11            if (i * i + j * j + k * k + i * j + j * k + j * i == n) {
    12              cnt++;
    13            }
    14          }
    15        }
    16      }
    17      printf("%d\n", cnt);
    18    }
    19    return 0;
    20  }
\end{verbatim}

これでは当然 TLE してしまう.ではどうすればいいか?

まず 8, 9 行目のあとに $i^2>n$, $i^2+j^2+ij>n$ ならループを抜けるように付け加えてみよう(ループが重たいので真面目に$i(i+2)+3$, $i(i+j+1)+j(j+1)+1$ で判定したほうが早くなるかも).

しかしこれでもまだ TLE.そこで $i \leq j \leq k$ になるようにすれば組み合わせはかなり減る.具体的には(枝刈りしていない状態で) $100^3$ から ${}_{100}\mathrm{C}_{3}$ に減る.つまり約 1/6 倍まで減らせるぞ!

このとき \verb|cnt| のインクリメントに注意!並べ替えを考慮していないので,$\#\set{i,j,k}=1$ なら 1,$\#\set{i,j,k}=2$ なら 3,$\#\set{i,j,k}=3$ なら 6 だけ増やすように場合分けしよう.$i \leq j \leq k$ から,$\#\set{i,j,k}=2 \Longrightarrow i < k$ であることをうまく利用しよう.

\subsection*{追記}
早い提出結果を眺めてみると,for文の条件に \verb|i * (i + j + k) + j * (j + k) + k <= n| が書いてあった.すげーかしこいとおもいました.

\end{document}
